\section{Birtday NP}
Michele's birthday is coming up and he is thinking about who to invite for the party. He has male friends, denoted by the set M, and female friends, denoted by the set F.
Between any two friends $x, y \in M \cup F$, there exists a score $w(x, y) \in \{0, 1\}$. The score tells us whether the two friends like each other or not (0 meaning that they do not like each other, 1 meaning that they like each other). For the party to be successful we have to consider the
following two constraints.
\begin{itemize}
	\item Michele would like to maximize the number of liked guests over the total number of
	invited guests. Formally, let $I \subseteq M \cup F$ be the invited guests. Then we want to maximize
	$\frac{1}{|I|} \sum_{x,y \in I} w(x,y)$
	\item Michele insists upon having an equal number of female and male guests. That is $|I \cap M| =
	|I \cap F|$.
\end{itemize}
\textbf{Goal:} Show that this problem is NP-complete.

\subsection{Proof}
\subsubsection{Formalization as graph problem}
We can represent each friend as a node in an undirected graph G. We also add an edge between two nodes \textit{x} and \textit{y} if and only if w(x,y) is equal to 1. In this case the function we want to maximize is exactly the density \cite{Dense subgraph} of a subgraph of G.

\subsubsection{Overview}
First of all let's notice that without the second constraint (number of males equal to number of females) the problem can be solved in polynomial time \cite{Dense subgraph}, since the first requirement is asking for the densest subgraph of G. I am going to show that adding the second constraint, the problem can be reduced to K-clique problem, that is a well-known NP-complete problem. I am going to define a variant of the above problem, that can simplify my proof. Let's call A the original problem, B the K-clique problem, and C the following: given a number K, find a subgraph with density (at least) K and an equal number of M and F nodes.
Problem A can be trivially reduced to problem C, because fixing a number \={K} we can check if problem C has a solution. If yes, we can increment \={K}, otherwise we can decrement it. After a certain number of iterations (see next for how many exactly) the problem C finds a solution for the maximum K: this solution is what the problem A is asking for. How many steps are needed to find the max K? We can start fixing \={K} = $|V|$. If a solution is found, of course it is optimal for A; otherwise, we behalf \={K} and retry, thus adopting a sort of \textit{binary search of K}, in the sense that we proceed in this way to find in a logarithmic (in function of $|V|$) number of steps the right K. So, finding a solution for problem A with this method simply increases the total cost of a factor $O(log|V|)$. It's time to proof that C can be reduced to B. If we are able to do this, we can conclude that also A can be reduced to B.
\subsubsection{NP certifier}
We can easily verify if an instance S is a solution of problem C. Indeed, we can verify if the number of males is equal to the number of females, and this can be done simply looking at the nodes of S. If we also count the number of edges in S, we can compute the density of S and check if it is equal to K. Now we can focus on the reduction.
\subsubsection{Reduction: Step 1}
Problem B asks for a clique of size $K_B$. Let's label all the nodes of G as males (the M set). At this point we can add a clique (we forge it, adding new nodes and all the necessary edges to the graph) of size $K_B$, all made up by females (i.e. the F set). This can be done in polynomial time, since we just have to add $K_B \leq |V|$ nodes and a number of edges that is a polynomial function of $|V|$ (quadratic, to be precise). 

\subsubsection{Reduction: Step 2}
Let's fix $K_C$ = $\frac{K_B-1}{2}$. Let's call ALG the algorithm that solves the problem C, and be SOL its solution.
At this point we can run ALG in order to find a solution SOL with density $K_C$. The solution, if any, \textbf{must} contain the whole clique of size $K_B$ of females (we added previously) and a clique of size $K_B$ of males. Remembering that the number of edges in a clique of n nodes is $\frac{n(n-1)}{2}$, the density of such a subgraph is, how we can expect, equal to $\frac{\frac{K_B(K_B - 1)}{2} + \frac{K_B(K_B - 1)}{2}}{2 \cdot K_B}  = \frac{K_B(K_B - 1)}{2 \cdot K_B} = \frac{K_B - 1}{2}$, that is exactly the number we fixed as $K_C$.
\newline
Is it possible that the solution found is not the union of two such cliques? Let's consider some cases.
Of course the number of male and female nodes must be the same, otherwise this is not a solution for problem C. Moreover, we know that the maximum number of female nodes is $K_B$, since we forged them at step 1. This means the that also the maximum number of male nodes in SOL must be equal to $K_B$.
Imagine that SOL is a clique with density $K_C$ and a number of male nodes strictly lower than $K_B$. This means that also the number of female nodes in SOL is strictly lower than $K_B$. But this leads to a density strictly lower than $K_C$, thus resulting in an absurd. And we also find an absurd, if we say that SOL is not the union of two cliques, and in particular, the male nodes do not form a clique, because the number of edges in this case would be strictly lower than what we are expecting, with a density strictly lower than $\frac{K_B -1}{2}$.
\subsubsection{Reduction: Step 3}
Problem B now can be solved, simply deleting from SOL all the females (forged) nodes, thus obtaining a clique of size $K_B$, all made up by male nodes of course. This step, as step 1, requires polynomial time. In particular, the cost is bounded to $O(|K_B|) \leq O(|V|)$ to delete nodes and becomes quadratic if we also remove the edges.

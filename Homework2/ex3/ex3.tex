\section{The "k min-cut" problem}
Let $F^*$ be an optimal solution for the problem and let $F^*_i$ be the cut that separates in the optimal solution $s_i$ from the other nodes. Since $F_i$ is a minimum cut for $s_i$ we have that: 
\[
\sum_{e \in F_i} c_e \leq \sum_{e \in F^*_i} c_e
\]
The cost of our solution, instead, is at most:
\[
\sum_{i=1}^{k} \sum_{e \in F_i} c_e \leq \sum_{i=1}^{k} \sum_{e \in F^*_i} c_e
\]
Since each edge in an optimal solution $F^*$ can be present in at most 2 different $F^*_i$, we have that our solution is bounded by:
\[
\sum_{i=1}^{k} \sum_{e \in F_i} c_e \leq \sum_{i=1}^{k} \sum_{e \in F^*_i} c_e \leq 2 \cdot \sum_{e \in F^*} c_e \leq 2 \cdot OPT
\]
and this shows the 2-approximation of the proposed algorithm.
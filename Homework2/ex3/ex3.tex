\section{The "k min-cut" problem}

Let $F^*$ be an optimal solution for the problem and let $F^*_i$ be the isolating cut in the optimal solution for $s_i$. Since $F_i$ is a minimum cut for $s_i$, 
\[
\sum_{e \in F_i} c_e \leq \sum_{e \in F^*_i} c_e
\]
The cost of our solution is at most
\[
\sum_{i=1}^{k} \sum_{e \in F_i} c_e \leq \sum_{i=1}^{k} \sum_{e \in F^*_i} c_e
\]
Since each edge in an optimal solution $F*$ can be present in at most 2 different $F^*_i$, we have that our solution is bounded by:
\[
\sum_{i=1}^{k} \sum_{e \in F_i} c_e \leq \sum_{i=1}^{k} \sum_{e \in F^*_i} c_e \leq 2 \cdot \sum_{e \in F*} c_e \leq 2 \cdot OPT
\]
and this shows the 2-approximation.
\section{Valerio and Set Cover}
Given a set A of required skills, a set S of all the available people, where each person is represented as a set of skills $S_j \subseteq A$, we can formulate the Set Cover with Redundancies problem using the following ILP:
\begin{align*}
\min&\sum_{S_j \in S} c(S_j) \cdot x_j \\
&\sum_{S_j | A_i \in S_j} x_j \geq 3, & \forall A_i \in A \\
&x_j \in \{0, 1\}, & \forall S_j \in S
\end{align*}
I am going to show a variant of the randomized rounding applied to set cover, starting from \cite{Vazirani}.
In order to build a randomized approximation consider the associated LP problem where $x^*_j \in [0, 1]$. The LP solution is a vector $x^*$ of real values.
For each set $S_j \in S$, pick $S_j$ with probability $x^*_j$\footnote{to be more precise $\min(x^*_j, 1)$}, the entry corresponding to $S_j$ in $x^*$. Let C be the collection of sets picked. The expected cost of C is
\[
E[c(C)] = \sum_{S_j \in S} Pr[S_j \text{ is picked}] \cdot c(S_j) = \sum_{S_j \in S} x^*_j \cdot c(S_j) = OPT_f.
\]
Next, let us compute the probability that a skill $a \in U$ is covered at least 3 times by C. Suppose that \textit{a} occurs in $k \geq 3$ (otherwise the problem has no solution) sets of S. Let the probabilities associated with these sets be $p_1, ..., p_k.$ Since \textit{a} is fractionally covered in the optimal solution, $\sum_{i=1}^{k} p_i \geq 3.$ The probability that \textit{a} is covered by \textit{C} is minimized when each of the $p_i$ is equal to $\frac{3}{k}.$ Thus,
\[
Pr[a \text{ is covered}] \geq  1 - \sum_{i=0}^{2} \binom{k}{i} (1-\frac{3}{k})^{k-i} = 1 - (1-\frac{3}{k})^k - 3 \cdot (1-\frac{3}{k})^{k-1} - \frac{9}{2}\cdot (1-\frac{3}{k})^{k-2}
\]
and we can bound this:
\[
Pr[a \text{ is covered}] \geq 1 - e^{-3} - 3e^{-3} - \frac{9}{2} e^{-3} \geq 1- e^{-\frac{5}{6}}
\]
To get a complete set cover with the redundancies, independently pick $\frac{6}{5}d \log n$ such subcollections, and compute their union, say C', where d is a constant such that:
$({e^{-\frac{5}{6}}})^{\frac{6}{5} d\log n} \leq \frac{1}{4n}$. Clearly we have that:
\[
Pr[a \text{ is not covered}] \leq \frac{1}{4n}
\]
Summing up all skills \textit{a}:
\[
Pr[C' \text{ is not a valid solution}] \leq n \cdot \frac{1}{4n} = \frac{1}{4}
\]
Clearly we have that:
\[
E[c(C')] \leq \frac{6}{5} \cdot OPT_f \cdot d \log n
\]
For Markov we can write:
\[
Pr[c(C') \geq OPT_f \cdot 4\cdot \frac{6}{5} \log n] \leq \frac{1}{4}
\]
and implies that
\[
Pr[C' \text{ is valid and has cost } \leq OPT_f \cdot 4\cdot \frac{6}{5}] \geq \frac{1}{2}
\]
If the above procedure fails to find a \textit{good}\footnote{where good means that it is a valid cover with redundancies and its cost is bounded by the expression defined above} solution, we can repeat the entire process one more time. The expected number of repetitions is at most 2.
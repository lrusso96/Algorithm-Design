\section{Cristina and DNA}
Let define a factorization \textit{f} of the string \textit{D} as an ordered multiset $\{g_1, g_2, ..., g_m\}$ with $g_i \in G$, such that the concatenation of $g_1, ..., g_m$ produces the string \textit{D}. Let \textit{F} be the set of all possible factorizations of \textit{D} and $F_g:= \{f \in F | g \in f\}$ be the set of all factorizations that contain the gene \textit{g}. We need $|G|$ boolean variables $x_g$, set to 1 if gene $g \in G$ is used to produce D and $|F|$ boolean variables $y_f$, indicating whether the factorization $f \in F$ is used or not. This would lead to an exponential number of variables! For this reason we can use an other approach \cite{Chris}: the factorizations of D are modeled as paths in the substring graph of D; in essence, its directed edges $(i,j)$ correspond to substring intervals, and the $|D| = n$ nodes to positions between characters. The following ILP formulation minimizes the cost of a path from source to sink with a unitary flow:
\begin{align*}
&\text{min} && \sum_{k=1}^{m} w_k \cdot x_k \\
&\text{s.t.} && x_{k_{|g_k = D[i:j]}} -z_{ij} \geq 0, \quad \forall (i,j) \in E\\
&&& \sum_{(i,j) \in \delta^+(0)} z_{ij} = 1 \\
&&& \sum_{(i,j) \in \delta^-(v)} z_{ij} = \sum_{(i,j) \in \delta^+(v)} z_{ij}, \quad \forall v \in \{1,..., n-1 \} \\
&&& x_k, z_{ij} \in \{0, 1\}, \quad \forall k \in \{1, ..., m\}, \forall (i,j) \in E
\end{align*}
After relaxing the problem to LP, using real variables $\geq 0$ and not boolean ones, we can compute the dual problem.
\begin{align*}
&\text{max} && b_0 \\
&\text{s.t.} && \sum_{(i,j) | g_k = D[i:j]} a_{ij} \geq w_k, \quad \forall k \in \{1,...,m\}\\
&&& a_{ij} \geq b_i - b_j, \quad \forall (i,j) \in E \\
&&& a_{ij} \geq 0, \quad \forall (i,j) \in E \\
\end{align*}
It is a maximization problem (the primal is a minimization one) with $n + |E|$ real variables deriving from the number of constraints of the primal, but not all of them are bounded (due to the equalities!); moreover, it has $m + |E|$ constraints due to the number of variables (both bounded and not) in the primal problem.
